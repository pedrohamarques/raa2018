\documentclass[raa-page]{lccv18}
%------------------------------------------------------------------------------
% LaTeX TEMPLATE FOR RAA 2018
%------------------------------------------------------------------------------
% DO NOT RENAME THIS FILE !!!!!!!

% REMARKS:
%------------------------------------------------------------------------------
% - USE AT LEAST 1 (ONE) AND AT MOST 2 (TWO) FIGURES
% - USE, PREFERENTIALLY, VECTOR GRAPHICS (PDF) INSTEAD OF RASTER GRAPHICS (JPEG)
%   OR PNG-IMAGES WITH TRANSPARENCY IN HIGH QUALITY
% - THE DOCUMENT MUST BE GENERATED ON A SINGLE PAGE


% FILL IN BELOW WITH YOUR WORK INFORMATION
\begin{document}
    
    % TITLE (DO NOT USE UPPERCASE)
    %------------------------------------------------------
    \settitle{A Study of Material Point Method applied to Dynamic Two-dimensional Problems}

    % AUTHORS' INFORMATION
    %------------------------------------------------------
    \setmainauthor{Pedro Henrique de Almeida Marques}
    \setmainauthoremail{pedro.marques@ctec.ufal.br}
    \setotherauthor{Tiago Peixoto da Silva Lôbo}
    \setotherauthor{Adeildo Soares Ramos Júnior}

    % TYPE OF WORK
    %------------------------------------------------------
    %\settype{Research \& Development Project}
    % \settype{Research \& Development Activity}
    % \settype{Research Beginner}
     \settype{Undergraduate Thesis}
    % \settype{Master's Thesis}
    % \settype{Doctoral Thesis}

    % DURATION AND STATUS
    %------------------------------------------------------
    \setduration{4 months}
    \setstatus{In progress}
    % \setstatus{Completed}

    % COURSE (REQUIRED FOR STUDENTS)
    %------------------------------------------------------
     \setcourse{Civil Engineering}
    % \setcourse{Chemical Engineering}
    % \setcourse{Petroleum Engineering}
    % \setcourse{Computer Science}

    % PARTNERS: CONTRIBUTOR LABS OR GROUPS (OPTIONAL)
    %------------------------------------------------------
    % \setpartners{CENPES/PETROBRAS}
    % \setpartners{LABGEO/UFAL}
    % \setpartners{IPT/USP}

    % FUNDING AGENCIES (OPTIONAL)
    %------------------------------------------------------
    \setfunding{CENPES/PETROBRAS}
    % \setfunding{CNPq}
    % \setfunding{FAPEAL}


    % INTRODUCTION TEXT AND FIGURE
    %------------------------------------------------------
    \setintro{Understanding how landslides occur and their aftermath is crucial to ensure that the submerged structures for hydrocarbon exploration are being kept safe. By studying how these impacts are transmitted to gasoducts, cables, risers, etc we aim to aid in the design of these structures. Therefore, a recommended method to simulate these large deformations is the Material Point Method (MPM). By discretizing a body in several particles containing properties such as position, velocity and stress and by utilizing temporal integration to define changes of those properties over the time, the MPM presents itself as a good candidate to simulate landslides and their interaction with structures.}
    
    %  \setintro{In the context of hydrocarbon exploration, the design of well liners is a crucial step, given their important structural and operational functions. The formulations recommended by standards have deterministic basis, not considering uncertainties associated to design parameters. It is proposed the probabilistic analysis of resistance models of pipes and connections, with special interest in the integrated safety evaluation, considering different modes of propagation of system failure.}
    %------------------------------------------------------
    \setintrofigure{intro.jpg}{height=5cm}

    % METHODOLOGY TEXT AND FIGURE
    %------------------------------------------------------
    \setmethodology{To find the numerical solution of the problem, the MPM utilizes a numerical integrator that is executed until the final time of simulation is achieved. A solution step is represented in the figure on the left: the body is discretized in a finite number particles that moves through a fixed background mesh, transmitting the information to the nodes where the forces are calculated. Then, this new information is transmitted to the particles where their position and velocity are updated, repeating the process until the end of the simulation. Furthermore, it avoids mesh distortion, as the mesh remains unchanged at the end of each step. This is one of the characteristics that makes the MPM suited for landslide simulation. }
    %------------------------------------------------------
    \setmethodologyfigure{mpmsteps.png}{width=6cm}

    % RESULTS TEXT AND FIGURE
    %------------------------------------------------------
   \setresults{ The main goal of this work is to simulate considering, as a first approximation, the soil as a linear elastic media, comparing the results obtained with the simplified model to the ones found on the literature, which uses elastoplastic, or even viscoelastoplastic models. Moreover, the scope of this work includes: the validation of the numerical program using analytical results; a parametric study of landslides (different slopes, \textit{geometry of initial mass}, different Elastic Moduli and Poisson coefficients); runout velocity; \textit{reach?}.}
    
    % \setresults{The developed models are coded into a \textit{web} repository that groups tools for well design. The system is capable of analyzing multiple user-created loading scenarios associated with a specific depth andincluding internal and external pressures, axial force, torque and \textit{dog leg}, with temperature effects. The failure mode safety factors are compared to the associated failure probabilities.}
    %------------------------------------------------------
    % \setresultsfigure{pdf-conjunta.png}{width=8cm}

    
    % DO NOT EDIT HERE !!!!!!!
    %------------------------------------------------------
    \createRAApage

\end{document}
