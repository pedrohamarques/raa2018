\documentclass[raa-page]{lccv18}
%------------------------------------------------------------------------------
% LaTeX TEMPLATE FOR RAA 2018
%------------------------------------------------------------------------------
% DO NOT RENAME THIS FILE !!!!!!!

% REMARKS:
%------------------------------------------------------------------------------
% - USE AT LEAST 1 (ONE) AND AT MOST 2 (TWO) FIGURES
% - USE, PREFERENTIALLY, VECTOR GRAPHICS (PDF) INSTEAD OF RASTER GRAPHICS (JPEG)
%   OR PNG-IMAGES WITH TRANSPARENCY IN HIGH QUALITY
% - THE DOCUMENT MUST BE GENERATED ON A SINGLE PAGE


% FILL IN BELOW WITH YOUR WORK INFORMATION
\begin{document}
    
    % TITLE (DO NOT USE UPPERCASE)
    %------------------------------------------------------
    \settitle{A Study of Material Point Method applied to Dynamic Two-dimensional Problems}

    % AUTHORS' INFORMATION
    %------------------------------------------------------
    \setmainauthor{Pedro Henrique de Almeida Marques}
    \setmainauthoremail{pedro.marques@ctec.ufal.br}
    \setotherauthor{Tiago Peixoto da Silva Lôbo}
    \setotherauthor{Adeildo Soares Ramos Júnior}

    % TYPE OF WORK
    %------------------------------------------------------
    %\settype{Research \& Development Project}
    % \settype{Research \& Development Activity}
    % \settype{Research Beginner}
     \settype{Undergraduate Thesis}
    % \settype{Master's Thesis}
    % \settype{Doctoral Thesis}

    % DURATION AND STATUS
    %------------------------------------------------------
    \setduration{4 months}
    \setstatus{In progress}
    % \setstatus{Completed}

    % COURSE (REQUIRED FOR STUDENTS)
    %------------------------------------------------------
     \setcourse{Civil Engineering}
    % \setcourse{Chemical Engineering}
    % \setcourse{Petroleum Engineering}
    % \setcourse{Computer Science}

    % PARTNERS: CONTRIBUTOR LABS OR GROUPS (OPTIONAL)
    %------------------------------------------------------
    % \setpartners{CENPES/PETROBRAS}
    % \setpartners{LABGEO/UFAL}
    % \setpartners{IPT/USP}

    % FUNDING AGENCIES (OPTIONAL)
    %------------------------------------------------------
    \setfunding{CENPES/PETROBRAS}
    % \setfunding{CNPq}
    % \setfunding{FAPEAL}


    % INTRODUCTION TEXT AND FIGURE
    %------------------------------------------------------
    \setintro{Understanding how landslides occur and their aftermath is crucial to ensure the submerged structures for hydrocarbon exploration are being kept safe by studying how these impacts are transmitted to gasoducts, cables, risers, etc. Therefore, a recommended method to simulate these large deformations is the Material Point Method. By discretizing a body in several particles containing properties such as position, velocity and stress and by utilizing temporal integration to define changes of those properties over the time, the MPM presents itself as a good candidate to simulate landslides and their interaction with structures.}
    

    %------------------------------------------------------
    % the figure is in portuguese, and with the text way too small to read. Increase the font size, and add the description of the figure in the text. This figure aims to explain how the MPM works. I suggest to move this figure to the methodology, where you can describe how the MPM works. Here you can add a figure of a landslide, explaining a little better what it is and how it affect the structures on its path.
    \setintrofigure{PRONTO.png}{height=6.5cm}

    % METHODOLOGY TEXT AND FIGURE
    %------------------------------------------------------
   %Change the focus of this section to the way MPM works. Explain the discretization method (material points that carries the information moving thorugh a fixed background mesh where the forces and calculated), explain that it avoids mesh distortion, so its suited for landslide simulation. Explain that you aim to simulate, as a first approximation, the soil as a linear elastic media, and compare your results to the ones found on the literature (which uses elastoplastic, or even viscoelastoplastic models).
    \setmethodology{The study of the weak formulation of Finite Element Method for the implementation of an algorithm to solve a problem of landslide through a linear elastic behavior, starting with the development of codes to determine numerical solutions using 1-D and 2-D linear basis functions. }
    %------------------------------------------------------
    \setmethodologyfigure{graf1d.png}{width=8cm}

    % RESULTS TEXT AND FIGURE
    %------------------------------------------------------
    %focus on what you aim to achieve, the validation of the numerical program, the simulation of the landslide, the parametric study. As you are still getting results, this will be expected results rather than results (you dont need to change the name of the section, i am just saying that you should write it as expected results).
    \setresults{Determination of the extent of the landslide using different types of soil and how occurs the redistribution of stresses due to the abrupt decrease in strength of the surface. Furthermore, the change of potential and kinetic energies during the slides and the \textit{runoff} velocity of the particles along different simulations will be determined.}
    

    %------------------------------------------------------
    % \setresultsfigure{pdf-conjunta.png}{width=8cm}

    
    % DO NOT EDIT HERE !!!!!!!
    %------------------------------------------------------
    \createRAApage

\end{document}
