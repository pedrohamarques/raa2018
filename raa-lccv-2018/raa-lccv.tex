\documentclass[raa-page]{lccv18}
%------------------------------------------------------------------------------
% LaTeX TEMPLATE FOR RAA 2018
%------------------------------------------------------------------------------
% DO NOT RENAME THIS FILE !!!!!!!

% REMARKS:
%------------------------------------------------------------------------------
% - USE AT LEAST 1 (ONE) AND AT MOST 2 (TWO) FIGURES
% - USE, PREFERENTIALLY, VECTOR GRAPHICS (PDF) INSTEAD OF RASTER GRAPHICS (JPEG)
%   OR PNG-IMAGES WITH TRANSPARENCY IN HIGH QUALITY
% - THE DOCUMENT MUST BE GENERATED ON A SINGLE PAGE


% FILL IN BELOW WITH YOUR WORK INFORMATION
\begin{document}
    
    % TITLE (DO NOT USE UPPERCASE)
    %------------------------------------------------------
    \settitle{Structural Reliability applied to Probabilistic Analysis of Well Coatings}

    % AUTHORS' INFORMATION
    %------------------------------------------------------
    \setmainauthor{Pedro Henrique de Almeida Marques}
    \setmainauthoremail{william@lccv.ufal.br}
    \setotherauthor{Eduardo T. de Lima Jr.}
    \setotherauthor{João Paulo L. Santos}
    \setotherauthor{Lucas P. de Gouveia}
    \setotherauthor{Rodrigo B. Paes}
    \setotherauthor{Willy C. Tiengo}
    
    % TYPE OF WORK
    %------------------------------------------------------
    \settype{Research \& Development Project}
    % \settype{Research \& Development Activity}
    % \settype{Research Beginner}
    % \settype{Undergraduate Thesis}
    % \settype{Master's Thesis}
    % \settype{Doctoral Thesis}

    % DURATION AND STATUS
    %------------------------------------------------------
    \setduration{36 months}
    \setstatus{In progress}
    % \setstatus{Completed}

    % COURSE (REQUIRED FOR STUDENTS)
    %------------------------------------------------------
    % \setcourse{Civil Engineering}
    % \setcourse{Chemical Engineering}
    % \setcourse{Petroleum Engineering}
    % \setcourse{Computer Science}

    % PARTNERS: CONTRIBUTOR LABS OR GROUPS (OPTIONAL)
    %------------------------------------------------------
    % \setpartners{CENPES/PETROBRAS}
    % \setpartners{LABGEO/UFAL}
    % \setpartners{IPT/USP}

    % FUNDING AGENCIES (OPTIONAL)
    %------------------------------------------------------
    \setfunding{CENPES/PETROBRAS}
    % \setfunding{CNPq}
    % \setfunding{FAPEAL}


    % INTRODUCTION TEXT AND FIGURE
    %------------------------------------------------------
    \setintro{In the context of hydrocarbon exploration, the design of well liners is a crucial step, given their important structural and operational functions. The formulations recommended by standards have deterministic basis, not considering uncertainties associated to design parameters. It is proposed the probabilistic analysis of resistance models of pipes and connections, with special interest in the integrated safety evaluation, considering different modes of propagation of system failure.}
    %------------------------------------------------------
    \setintrofigure{wellscheme.png}{height=6.5cm}

    % METHODOLOGY TEXT AND FIGURE
    %------------------------------------------------------
    \setmethodology{The geometry specifications of the pipes and connections, as well as their resistance models, are defined in specific API standards. Reliability models are formulated based on the FORM (First Order Reliability Method) method, using production statistics data related to mechanical dimensions and properties of tubular. Load scenarios are mapped from failure trees to identify critical trajectories for coating integrity. System reliability theory is applied to global failure assessment.}
    %------------------------------------------------------
    \setmethodologyfigure{pdf-conjunta.png}{width=8cm}

    % RESULTS TEXT AND FIGURE
    %------------------------------------------------------
    \setresults{The developed models are coded into a \textit{web} repository that groups tools for well design. The system is capable of analyzing multiple user-created loading scenarios associated with a specific depth andincluding internal and external pressures, axial force, torque and \textit{dog leg}, with temperature effects. The failure mode safety factors are compared to the associated failure probabilities.}
    %------------------------------------------------------
    % \setresultsfigure{pdf-conjunta.png}{width=8cm}

    
    % DO NOT EDIT HERE !!!!!!!
    %------------------------------------------------------
    \createRAApage

\end{document}
